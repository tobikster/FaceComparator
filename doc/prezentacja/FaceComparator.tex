\documentclass{beamer}

\usepackage{polski}
\usepackage[utf8]{inputenc}

\usetheme{CambridgeUS}

\title{Face Comparator}
\author[PT MW]{
	inż. Paweł Tobiszewski, 179169\\*
	inż. Marcin Ważeliński, 179151
}
\institute[PWr]{Wydział Informatyki i Zarządzania, Politechnika Wrocławska}
\date{13 czerwca 2013}

\begin{document}

\begin{frame}
	\titlepage
\end{frame}

\section{Przedstawienie problemu}
\begin{frame}
\frametitle{Przedstawienie problemu}
	\begin{itemize}
	\item Problem --- porównywanie twarzy
	\item Decyzje --- zdjęcia twarzy
	\item Kryteria --- podawane przez Użytkownika
	\item Do rozwiązania wykorzystano metodę AHP
	\end{itemize}
\end{frame}

\section{Przykład problemu}
\begin{frame}
\frametitle{Przykład - wstęp}
\end{frame}

\subsection{Definicja problemu}
\begin{frame}
\frametitle{Macierze porównań kryteriów}
	\begin{columns}
		\begin{column}{0.5\textwidth}
			\begin{table}
			\caption{Porównanie kryteriów}
				\begin{tabular}{c|c|c|c}
					&	$K_1$	&	$K_2$	&	$K_3$\\ \hline
				$K_1$	&	$1.0$	&	$1.0$	&	$7.0$\\ \hline
				$K_2$	&	$1.0$	&	$1.0$	&	$3.0$\\ \hline
				$K_3$	&	$0.14$	&	$0.33$	&	$1.0$
				\end{tabular}
			\end{table}
			\begin{table}
			\caption{Kryterium 1}
				\begin{tabular}{c|c|c|c}
					&	$D_1$	&	$D_2$	&	$D_3$\\ \hline
				$D_1$	&	$1.0$	&	$0.2$	&	$5.0$\\ \hline
				$D_2$	&	$5.0$	&	$1.0$	&	$7.0$\\ \hline
				$D_3$	&	$0.2$	&	$0.14$	&	$1.0$
				\end{tabular}
			\end{table}
		\end{column}
		\begin{column}{0.5\textwidth}
			\begin{table}
			\caption{Kryterium 2}
				\begin{tabular}{c|c|c|c}
					&	$D_1$	&	$D_2$	&	$D_3$\\ \hline
				$D_1$	&	$1.0$	&	$3.0$	&	$9.0$\\ \hline
				$D_2$	&	$0.33$	&	$1.0$	&	$3.0$\\ \hline
				$D_3$	&	$0.11$	&	$0.33$	&	$1.0$
				\end{tabular}
			\end{table}
			\begin{table}
			\caption{Kryterium 3}
				\begin{tabular}{c|c|c|c}
					&	$D_1$	&	$D_2$	&	$D_3$\\ \hline
				$D_1$	&	$1.0$	&	$0.2$	&	$0.11$\\ \hline
				$D_2$	&	$5.0$	&	$1.0$	&	$0.14$\\ \hline
				$D_3$	&	$9.0$	&	$7.0$	&	$1.0$
				\end{tabular}
			\end{table}
		\end{column}
	\end{columns}
\end{frame}

\subsection{Normalizacja macierzy}
\begin{frame}
\frametitle{Etap 1 --- normalizacja macierzy}
Aby znormalizować macierze, należy najpierw policzyć sumy w kolumnach, a następnie każdą wartość komórki macierzy podzielić przez sumę z odpowiadającej jej kolumny.
Przykład dla macierzy porównań kryteriów:
	\begin{columns}
		\begin{column}{0.4\textwidth}
			\begin{table}
				\begin{tabular}{c|c|c|c}
					&	$D_1$	&	$D_2$	&	$D_3$\\ \hline
				$D_1$	&	$1.0$	&	$1.0$	&	$7.0$\\ \hline
				$D_2$	&	$1.0$	&	$1.0$	&	$3.0$\\ \hline
				$D_3$	&	$0.14$	&	$0.33$	&	$1.0$\\ \hline\hline
				$c_0$	&	$2.14$	&	$2.33$	&	$11.0$
				\end{tabular}
			\end{table}
		\end{column}
		\begin{column}{0.1\textwidth}
			$\rightarrow$
		\end{column}
		\begin{column}{0.4\textwidth}
			\begin{table}
				\begin{tabular}{c|c|c|c}
					&	$D_1$	&	$D_2$	&	$D_3$\\ \hline
				$D_1$	&	$0.47$	&	$0.43$	&	$0.64$\\ \hline
				$D_2$	&	$0.47$	&	$0.43$	&	$0.27$\\ \hline
				$D_3$	&	$0.07$	&	$0.14$	&	$0.09$
				\end{tabular}
			\end{table}
		\end{column}
	\end{columns}
	$c_0$ oznacza wektor sum
\end{frame}

\begin{frame}[allowframebreaks]
	Kryterium 1
	\begin{columns}
		\begin{column}{0.4\textwidth}
			\begin{table}
				\begin{tabular}{c|c|c|c}
					&	$D_1$	&	$D_2$	&	$D_3$\\ \hline
				$D_1$	&	$1.0$	&	$0.2$	&	$5.0$\\ \hline
				$D_2$	&	$5.0$	&	$1.0$	&	$7.0$\\ \hline
				$D_3$	&	$0.2$	&	$0.14$	&	$1.0$\\ \hline\hline
				$c_1$	&	$6.20$	&	$1.34$	&	$13.0$
				\end{tabular}
			\end{table}
		\end{column}
		\begin{column}{0.1\textwidth}
			$\rightarrow$
		\end{column}
		\begin{column}{0.4\textwidth}
			\begin{table}
				\begin{tabular}{c|c|c|c}
					&	$D_1$	&	$D_2$	&	$D_3$\\ \hline
				$D_1$	&	$0.16$	&	$0.15$	&	$0.38$\\ \hline
				$D_2$	&	$0.81$	&	$0.74$	&	$0.54$\\ \hline
				$D_3$	&	$0.03$	&	$0.11$	&	$0.08$
				\end{tabular}
			\end{table}
		\end{column}
	\end{columns}
	Kryterium 2
	\begin{columns}
		\begin{column}{0.4\textwidth}
			\begin{table}
				\begin{tabular}{c|c|c|c}
					&	$D_1$	&	$D_2$	&	$D_3$\\ \hline
				$D_1$	&	$1.0$	&	$0.2$	&	$5.0$\\ \hline
				$D_2$	&	$5.0$	&	$1.0$	&	$7.0$\\ \hline
				$D_3$	&	$0.2$	&	$0.14$	&	$1.0$\\ \hline\hline
				$c_2$	&	$6.20$	&	$1.34$	&	$13.0$
				\end{tabular}
			\end{table}
		\end{column}
		\begin{column}{0.1\textwidth}
			$\rightarrow$
		\end{column}
		\begin{column}{0.4\textwidth}
			\begin{table}
				\begin{tabular}{c|c|c|c}
					&	$D_1$	&	$D_2$	&	$D_3$\\ \hline
				$D_1$	&	$0.16$	&	$0.15$	&	$0.38$\\ \hline
				$D_2$	&	$0.81$	&	$0.74$	&	$0.54$\\ \hline
				$D_3$	&	$0.03$	&	$0.11$	&	$0.08$
				\end{tabular}
			\end{table}
		\end{column}
	\end{columns}
	\framebreak
	Kryterium 3
	\begin{columns}
		\begin{column}{0.4\textwidth}
			\begin{table}
				\begin{tabular}{c|c|c|c}
					&	$D_1$	&	$D_2$	&	$D_3$\\ \hline
				$D_1$	&	$1.0$	&	$0.2$	&	$5.0$\\ \hline
				$D_2$	&	$5.0$	&	$1.0$	&	$7.0$\\ \hline
				$D_3$	&	$0.2$	&	$0.14$	&	$1.0$\\ \hline\hline
				$c_3$	&	$6.20$	&	$1.34$	&	$13.0$
				\end{tabular}
			\end{table}
		\end{column}
		\begin{column}{0.1\textwidth}
			$\rightarrow$
		\end{column}
		\begin{column}{0.4\textwidth}
			\begin{table}
				\begin{tabular}{c|c|c|c}
					&	$D_1$	&	$D_2$	&	$D_3$\\ \hline
				$D_1$	&	$0.16$	&	$0.15$	&	$0.38$\\ \hline
				$D_2$	&	$0.81$	&	$0.74$	&	$0.54$\\ \hline
				$D_3$	&	$0.03$	&	$0.11$	&	$0.08$
				\end{tabular}
			\end{table}
		\end{column}
	\end{columns}
\end{frame}

\subsection{Wyznaczenie wektorów preferencji}
\begin{frame}
\frametitle{Etap 2 --- wyznaczenie wektorów preferencji}
Dla każdej ze znormalizowanych macierzy należy wyznaczyć wektor preferencji --- wyliczając średnie arytmetyczne wartości w wierszach macierzy.
	\begin{columns}
		\begin{column}{0.4\textwidth}
			\begin{table}
				\begin{tabular}{c|c|c|c}
					&	$D_1$	&	$D_2$	&	$D_3$\\ \hline
				$D_1$	&	$0.47$	&	$0.43$	&	$0.64$\\ \hline
				$D_2$	&	$0.47$	&	$0.43$	&	$0.27$\\ \hline
				$D_3$	&	$0.07$	&	$0.14$	&	$0.09$
				\end{tabular}
			\end{table}
		\end{column}
		\begin{column}{0.1\textwidth}
			$\rightarrow$
		\end{column}
		\begin{column}{0.1\textwidth}
			\begin{table}
				\begin{tabular}{c}
				$s_0$\\ \hline
				$0.51$\\
				$0.39$\\
				$0.1$
				\end{tabular}
			\end{table}
		\end{column}
	\end{columns}
\end{frame}

\begin{frame}
	\begin{columns}
		\begin{column}{0.4\textwidth}
			\begin{table}
				\begin{tabular}{c|c|c|c}
				$K_1$	&	$D_1$	&	$D_2$	&	$D_3$\\ \hline
				$D_1$	&	$0.16$	&	$0.15$	&	$0.38$\\ \hline
				$D_2$	&	$0.81$	&	$0.74$	&	$0.54$\\ \hline
				$D_3$	&	$0.03$	&	$0.11$	&	$0.08$
				\end{tabular}
			\end{table}
		\end{column}
		\begin{column}{0.1\textwidth}
			$\rightarrow$
		\end{column}
		\begin{column}{0.1\textwidth}
			\begin{table}
				\begin{tabular}{c}
				$s_0$\\ \hline
				$0.23$\\
				$0.70$\\
				$0.07$
				\end{tabular}
			\end{table}
		\end{column}
	\end{columns}
	\begin{columns}
		\begin{column}{0.4\textwidth}
			\begin{table}
				\begin{tabular}{c|c|c|c}
				$K_2$	&	$D_1$	&	$D_2$	&	$D_3$\\ \hline
				$D_1$	&	$0.16$	&	$0.15$	&	$0.38$\\ \hline
				$D_2$	&	$0.81$	&	$0.74$	&	$0.54$\\ \hline
				$D_3$	&	$0.03$	&	$0.11$	&	$0.08$
				\end{tabular}
			\end{table}
		\end{column}
		\begin{column}{0.1\textwidth}
			$\rightarrow$
		\end{column}
		\begin{column}{0.1\textwidth}
			\begin{table}
				\begin{tabular}{c}
				$s_0$\\ \hline
				$0.69$\\
				$0.23$\\
				$0.08$
				\end{tabular}
			\end{table}
		\end{column}
	\end{columns}
	\begin{columns}
		\begin{column}{0.4\textwidth}
			\begin{table}
				\begin{tabular}{c|c|c|c}
				$K_3$	&	$D_1$	&	$D_2$	&	$D_3$\\ \hline
				$D_1$	&	$0.16$	&	$0.15$	&	$0.38$\\ \hline
				$D_2$	&	$0.81$	&	$0.74$	&	$0.54$\\ \hline
				$D_3$	&	$0.03$	&	$0.11$	&	$0.08$
				\end{tabular}
			\end{table}
		\end{column}
		\begin{column}{0.1\textwidth}
			$\rightarrow$
		\end{column}
		\begin{column}{0.1\textwidth}
			\begin{table}
				\begin{tabular}{c}
				$s_0$\\ \hline
				$0.06$\\
				$0.19$\\
				$0.75$
				\end{tabular}
			\end{table}
		\end{column}
	\end{columns}
\end{frame}

\subsection{Wyznaczenie rankingu decyzji}
\begin{frame}[allowframebreaks]
\frametitle{Etap 3 - wyznaczenie rankingu decyzji}
Aby wyznaczyć ranking decyzji, należy pomnożyć macierz powstałą przez ,,sklejenie'' wektorów preferencji względem każdego z kryteriów przez wektor preferencji kryteriów:
$$ R=\left[c_1 c_2 c_3\right]\times[c_0] $$
$$
	R=
	\left[
	\begin{array}{ccc}
		c_1^{(1)}	&	c_2^{(1)}	&	c_3^{(1)}\\
		c_1^{(2)}	&	c_2^{(2)}	&	c_3^{(2)}\\
		c_1^{(3)}	&	c_2^{(3)}	&	c_3^{(3)}\\
	\end{array}
	\right]
	\times
	\left[
	\begin{array}{ccc}
		c_0^{(1)}\\
		c_0^{(2)}\\
		c_0^{(3)}\\
	\end{array}
	\right]
$$
\framebreak
$$
	R=
	\left[
	\begin{array}{ccc}
		0.23	&	0.69	&	0.06\\
		0.70	&	0.23	&	0.19\\
		0.07	&	0.08	&	0.75\\
	\end{array}
	\right]
	\times
	\left[
	\begin{array}{ccc}
		$0.51$\\
		$0.39$\\
		$0.1$
	\end{array}
	\right]
	=
	\left[
	\begin{array}{ccc}
		0.39\\
		0.46\\
		0.14\\
	\end{array}
	\right]
$$
W wyniku otrzymujemy wektor rankingu decyzji --- każdy wiersz odpowiada kolejnej decyzji.
Z wektora tego możemy odczytać, że najbardziej preferowaną twarzą (według zadanych macierzy porównań) jest twarz druga.
Kolejną --- twarz pierwsza, a najmniej odpowiada nam twarz trzecia.
\end{frame}

\subsection{Test spójności}
\begin{frame}
\frametitle{Etap 4 --- test spójności}
\end{frame}
\end{document}
