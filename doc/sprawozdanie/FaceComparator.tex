\documentclass[a4paper,notitlepage]{article}

\usepackage{polski}
\usepackage[utf8]{inputenc}
\usepackage[left=2cm,right=2cm,top=3cm,bottom=3cm]{geometry}
\usepackage{multirow}


\author{
	inż. Paweł Tobiszewski, 179769\\
	inż. Marcin Ważeliński, 179151
}
\title{
	Face Comparator --- sprawozdanie\\
	Systemy wspomagania decyzji - projekt\\
	czwartek, \hour{18}{55} -- \hour{20}{30}
}
\date{}

\newcommand{\hour}[2]{
	$#1^{\underline{#2}}$
}

\begin{document}
\maketitle

\section{Wstęp}
Podczas zajęć projektowych naszym zadaniem było zastosowanie metod poznanych na ćwiczeniach i w trakcie wykładu do zaimplementowania algorytmu wspomagającego rozwiązanie wybranego problemu decyzyjnego.
Wybranym przez nas problemem było wielokryterialne porównywanie twarzy, więc zastosowaliśmy metodę AHP.

\subsection{Przedstawienie problemu}
Aplikacja realizowana w trakcie projektu będzie pomagać rozwiązać problem wyboru najbardziej odpowiadającej użytkownikowi twarzy.
Kryteria pod względem których oceniane będą twarze poda sam użytkownik --- dla działania metody AHP ważna jest tylko ilość kryteriów, ich nazwy nie mają znaczenia dla działania algorytmu.

\section{Przykład zastosowania metody AHP do rozwiązania problemu wyboru twarzy}

Przykładowy problem wyboru twarzy:
\begin{itemize}
\item decyzje - 3 twarze: d1, d2, d3
\item kryteria - k1, k2, k3 (nazwy nie są ważne dla działania metody)
\end{itemize}

\begin{table}[!ht]
\centering
\caption{Macierz porównań kryteriów}
	\begin{tabular}{c|c|c|c}
		&	$K_1$	&	$K_2$	&	$K_3$\\ \hline
	$K_1$	&	$1.0$	&	$1.0$	&	$7.0$\\ \hline
	$K_2$	&	$1.0$	&	$1.0$	&	$3.0$\\ \hline
	$K_3$	&	$0.14$	&	$0.33$	&	$1.0$
	\end{tabular}
\end{table}
\begin{table}[!ht]
\centering
\caption{Macierz porównań decyzji względem pierwszego kryterium}
	\begin{tabular}{c|c|c|c}
		&	$D_1$	&	$D_2$	&	$D_3$\\ \hline
	$D_1$	&	$1.0$	&	$0.2$	&	$5.0$\\ \hline
	$D_2$	&	$5.0$	&	$1.0$	&	$7.0$\\ \hline
	$D_3$	&	$0.2$	&	$0.14$	&	$1.0$
	\end{tabular}
\end{table}
\begin{table}[!ht]
\centering
\caption{Macierz porównań decyzji względem drugiego kryterium}
	\begin{tabular}{c|c|c|c}
		&	$D_1$	&	$D_2$	&	$D_3$\\ \hline
	$D_1$	&	$1.0$	&	$3.0$	&	$9.0$\\ \hline
	$D_2$	&	$0.33$	&	$1.0$	&	$3.0$\\ \hline
	$D_3$	&	$0.11$	&	$0.33$	&	$1.0$
	\end{tabular}
\end{table}
\begin{table}[!ht]
\centering
\caption{Macierz porównań decyzji względem trzeciego kryterium}
	\begin{tabular}{c|c|c|c}
		&	$D_1$	&	$D_2$	&	$D_3$\\ \hline
	$D_1$	&	$1.0$	&	$0.2$	&	$0.11$\\ \hline
	$D_2$	&	$5.0$	&	$1.0$	&	$0.14$\\ \hline
	$D_3$	&	$9.0$	&	$7.0$	&	$1.0$
	\end{tabular}
\end{table}

\pagebreak
\subsection{Etap 1 --- wyznaczenie wektorów preferencji}
Metoda AHP polega na wyliczeniu rankingu decyzji, opierając się na tzw. wektorach preferencji --- obliczanych dla każdej z powyższych macierzy.
Aby wyznaczyć wektor preferencji, należy najpierw policzyć sumy w kolumnach macierzy. Następnie sumy te wykorzystane zostaną do znormalizowania macierzy.
Wektor preferencji wyznaczony jest poprzez uśrednienie wartości w wierszach znormalizowanych macierzy.

\begin{table}[!ht]
\centering
\caption{Wyznaczenie wektora preferencji dla macierzy porównań kryteriów}
	\begin{tabular}{ccccc}
		\begin{tabular}{c|c|c|c}
			&	$D_1$	&	$D_2$	&	$D_3$\\ \hline
		$D_1$	&	$1.0$	&	$1.0$	&	$7.0$\\ \hline
		$D_2$	&	$1.0$	&	$1.0$	&	$3.0$\\ \hline
		$D_3$	&	$0.14$	&	$0.33$	&	$1.0$\\
		\multicolumn{4}{c}{
			$c_1=
			\left[
				\begin{array}{ccc}
				6.20		&	1.34		&	13.0\\
				\end{array}
			\right]$
		}
		\end{tabular}
	&
	$\rightarrow$
	&
		\begin{tabular}{c|c|c|c}
			&	$D_1$	&	$D_2$	&	$D_3$\\ \hline
		$D_1$	&	$0.47$	&	$0.43$	&	$0.64$\\ \hline
		$D_2$	&	$0.47$	&	$0.43$	&	$0.27$\\ \hline
		$D_3$	&	$0.07$	&	$0.14$	&	$0.09$
		\end{tabular}
	&
	$\rightarrow$
	&
	$s_0=
	\left[
		\begin{array}{c}
		0.51\\
		0.39\\
		0.1\\
		\end{array}
		\right]$
	\end{tabular}
\end{table}
\begin{table}[!ht]
\centering
\caption{Wyznaczenie wektora sum dla macierzy porównań decyzji względem pierwszego kryterium}
	\begin{tabular}{ccccc}
		\begin{tabular}{c|c|c|c}
			&	$D_1$	&	$D_2$	&	$D_3$\\ \hline
		$D_1$	&	$1.0$	&	$0.2$	&	$5.0$\\ \hline
		$D_2$	&	$5.0$	&	$1.0$	&	$7.0$\\ \hline
		$D_3$	&	$0.2$	&	$0.14$	&	$1.0$\\
		\multicolumn{4}{c}{
			$c_1=
			\left[
				\begin{array}{ccc}
				6.20		&	1.34		&	13.0\\
				\end{array}
			\right]$
		}
		\end{tabular}
	&
	$\rightarrow$
	&
		\begin{tabular}{c|c|c|c}
			&	$D_1$	&	$D_2$	&	$D_3$\\ \hline
		$D_1$	&	$0.16$	&	$0.15$	&	$0.38$\\ \hline
		$D_2$	&	$0.81$	&	$0.74$	&	$0.54$\\ \hline
		$D_3$	&	$0.03$	&	$0.11$	&	$0.08$
		\end{tabular}
	&
	$\rightarrow$
	&
	$s_1=
	\left[
		\begin{array}{c}
		0.23\\
		0.70\\
		0.07\\
		\end{array}
	\right]$
	\end{tabular}
\end{table}
\begin{table}[!ht]
\centering
\caption{Wyznaczenie wektora preferencji dla macierzy porównań decyzji względem drugiego kryterium}
	\begin{tabular}{ccccc}
		\begin{tabular}{c|c|c|c}
			&	$D_1$	&	$D_2$	&	$D_3$\\ \hline
		$D_1$	&	$1.0$	&	$0.2$	&	$5.0$\\ \hline
		$D_2$	&	$5.0$	&	$1.0$	&	$7.0$\\ \hline
		$D_3$	&	$0.2$	&	$0.14$	&	$1.0$\\
		\multicolumn{4}{c}{
			$c_2=
			\left[
				\begin{array}{ccc}
				6.20		&	1.34		&	13.0\\
				\end{array}
			\right]$
		}
		\end{tabular}
	&
	$\rightarrow$
	&
		\begin{tabular}{c|c|c|c}
			&	$D_1$	&	$D_2$	&	$D_3$\\ \hline
		$D_1$	&	$0.16$	&	$0.15$	&	$0.38$\\ \hline
		$D_2$	&	$0.81$	&	$0.74$	&	$0.54$\\ \hline
		$D_3$	&	$0.03$	&	$0.11$	&	$0.08$
		\end{tabular}
	&
	$\rightarrow$
	&
	$s_2=
	\left[
		\begin{array}{c}
		0.69\\
		0.23\\
		0.08\\
		\end{array}
	\right]$
	\end{tabular}
\end{table}
\begin{table}[!ht]
\centering
\caption{Wyznaczenie wektora preferencji dla macierzy porównań decyzji względem trzeciego kryterium}
	\begin{tabular}{ccccc}
		\begin{tabular}{c|c|c|c}
			&	$D_1$	&	$D_2$	&	$D_3$\\ \hline
		$D_1$	&	$1.0$	&	$0.2$	&	$5.0$\\ \hline
		$D_2$	&	$5.0$	&	$1.0$	&	$7.0$\\ \hline
		$D_3$	&	$0.2$	&	$0.14$	&	$1.0$\\
		\multicolumn{4}{c}{
			$c_3=
			\left[
				\begin{array}{ccc}
				6.20		&	1.34		&	12.0\\
				\end{array}
			\right]$
		}
		\end{tabular}
	&
	$\rightarrow$
	&
		\begin{tabular}{c|c|c|c}
		$K_3$	&	$D_1$	&	$D_2$	&	$D_3$\\ \hline
		$D_1$	&	$0.16$	&	$0.15$	&	$0.38$\\ \hline
		$D_2$	&	$0.81$	&	$0.74$	&	$0.54$\\ \hline
		$D_3$	&	$0.03$	&	$0.11$	&	$0.08$
		\end{tabular}
	&
	$\rightarrow$
	&
	$s_3=
	\left[
		\begin{array}{c}
			0.06\\
			0.19\\
			0.75
		\end{array}
	\right]$
	\end{tabular}
\end{table}

\subsection{Wyznaczenie rankingu decyzji}
\end{document}
